\chapter{TOUGH2 grids}
\label{t2grids}

\section{Introduction}
The \texttt{t2grids} library in PyTOUGH contains classes and routines for manipulating TOUGH2 grids.  It can be imported using the command:

\begin{verbatim}
    from t2grids import *
\end{verbatim}

\section{\texttt{t2grid} objects}

The \texttt{t2grids} library defines a \texttt{t2grid} class, used for representing TOUGH2 grids.  This gives access via Python to the grid's rock types, blocks, connections and other parameters.

Normally a TOUGH2 grid is not created directly, but is either read from a TOUGH2 data file, or constructed from a \texttt{mulgrid} geometry object (see chapter \ref{mulgrids}) using the \texttt{fromgeo} method (see section \ref{t2gridmethods}).

Printing a \texttt{t2grid} object (e.g. \texttt{print grid}) displays a summary of information about the grid: how many rock types, blocks and connections it contains.

\subsection{Properties}

The main properties of a \texttt{t2grid} object are listed in Table \ref{tb:t2grid_properties}.  Essentially a \texttt{t2grid} object contains collections of blocks, rock types and connections, each accessible either by name or by index.  For example, block `AB 20' in a \texttt{t2grid} called \texttt{grid} is given by \texttt{grid.block[`AB 20']}.

Connections are slightly different from blocks or rock types, in that they are not named individually.  However, they can be accessed by the names of the blocks connected by the connection.  For example, the connection between blocks `aa 10' and `ab 10' in a \texttt{t2grid} called \texttt{grid} is given by \texttt{grid.connection[`aa 10',`ab 10']}.

The \texttt{rocktype\_frequencies} property gives information about how frequently each rock type is used (i.e. how many blocks use that rock type).  It returns a list of tuples, the first element of each tuple being the frequency of use, and the second element being a list of rock type names with that frequency.  The list is given in order of increasing frequency.

The \texttt{rocktype\_indices} property gives an \texttt{np.array} containing the index of the rocktype for each block in the grid.  This can be used to give a plot of rock types, in conjunction with the \texttt{mulgrid} methods \texttt{layer\_plot} or \texttt{slice\_plot}.

\begin{table}
  \begin{center}
    \begin{tabular}{|l|l|l|}
      \hline
      \textbf{Property} & \textbf{Type} & \textbf{Description}\\
      \hline
      \texttt{atmosphere\_blocks} & list & atmosphere blocks\\
      \texttt{blocklist} & list & blocks (by index)\\
      \texttt{block} & dictionary & blocks (by name)\\
      \texttt{block\_centres\_defined} & Boolean & whether block centres have been calculated\\
      \texttt{connectionlist} & list & connections (by index)\\
      \texttt{connection} & dictionary & connections (by tuples of block names)\\
      \texttt{num\_atmosphere\_blocks} & integer & number of atmosphere blocks\\
      \texttt{num\_blocks} & integer & number of blocks\\
      \texttt{num\_connections} & integer & number of connections\\
      \texttt{num\_rocktypes} & integer & number of rock types\\
      \texttt{num\_underground\_blocks} & integer & number of non-atmosphere blocks\\
      \texttt{rocktypelist} & list & rock types (by index)\\
      \texttt{rocktype} & dictionary & rock types (by name)\\
      \texttt{rocktype\_frequencies} & list of tuples & frequencies of rock types\\
      \texttt{rocktype\_indices} & \texttt{np.array} & index of rock type for each block\\
      \hline
    \end{tabular}
    \caption{Properties of a \texttt{t2grid} object}
    \label{tb:t2grid_properties}
  \end{center}
\end{table}

\subsection{Methods}
\label{t2gridmethods}

The main methods of a \texttt{t2grid} object are listed in Table \ref{tb:t2grid_methods}.  Details of these methods are given below.

\begin{table}
  \begin{center}
    \begin{tabular}{|l|l|p{65mm}|}
      \hline
      \textbf{Method} & \textbf{Type} & \textbf{Description}\\
      \hline
      \texttt{+} & \texttt{t2grid} & adds two grids together\\
      \texttt{add\_block} & -- & adds a block to the grid\\
      \texttt{add\_connection} & -- & adds a connection to the grid\\
      \texttt{add\_rocktype} & -- & adds a rock type to the grid\\
      \texttt{block\_index} & integer & returns index of a block with a specified name\\
      \texttt{calculate\_block\_centres} & -- & calculates geometrical centre of all blocks in the grid\\
      \texttt{check} & Boolean & checks grid for errors and optionally fixes them\\
      \texttt{clean\_rocktypes} & -- & deletes any unused rock types from the grid\\
      \texttt{connection\_index} & integer & returns index of a connection with a specified pair of names\\
      \texttt{copy\_connection\_directions} & -- & copies connection permeability directions from another grid\\
      \texttt{delete\_block} & -- & deletes a block from the grid\\
      \texttt{delete\_connection} & -- & deletes a connection from the grid\\
      \texttt{delete\_rocktype} & -- & deletes a rock type from the grid\\
      \texttt{empty} & -- & empties contents of grid\\
      \texttt{fromgeo} & \texttt{t2grid} & constructs a TOUGH2 grid from a \texttt{mulgrid} object\\
      \texttt{rocktype\_frequency} & integer & frequency of use of a particular rock type\\
      \texttt{sort\_rocktypes} & -- & sorts rock type list into alphabetical order by name\\
      \texttt{write\_vtk} & -- & writes grid to VTK file\\
      \hline
    \end{tabular}
    \caption{Methods of a \texttt{t2grid} object}
    \label{tb:t2grid_methods}
  \end{center}
\end{table}

\subsubsection{\texttt{+}}

Adds two grids \texttt{a} and \texttt{b} together (i.e. amalgamates them) to form a new grid \texttt{a+b}.  If any rock types, blocks or connections exist in both grids \texttt{a} and \texttt{b}, the value from \texttt{b} is used, so there are no duplicates.

\textbf{Parameters:}
\begin{itemize}
\item \textbf{a, b}: \texttt{t2grid}\\
  The two grids to be added together.
\end{itemize}

\subsubsection{\texttt{add\_block(\emph{block})}}

Adds a block to the grid.  If another block with the same name already exists, it is replaced.

\textbf{Parameters:}
\begin{itemize}
\item \textbf{block}: \texttt{t2block}\\
  Block to be added to the grid.
\end{itemize}

\subsubsection{\texttt{add\_connection(\emph{connection})}}

Adds a connection to the grid.  If another connection with the same column names already exists, it is replaced.

\textbf{Parameters:}
\begin{itemize}
\item \textbf{connection}: \texttt{t2connection}\\
  Connection to be added to the grid.
\end{itemize}

\subsubsection{\texttt{add\_rocktype(\emph{rock})}}

Adds a rock type to the grid.  If another rock type with the same name already exists, it is replaced.

\textbf{Parameters:}
\begin{itemize}
\item \textbf{rock}: \texttt{rocktype}\\
  Rock type to be added to the grid.
\end{itemize}

\subsubsection{\texttt{block\_index(\emph{blockname})}}

Returns the block index (in the \texttt{blocklist} list) of a specified block name.

\textbf{Parameters:}
\begin{itemize}
\item \textbf{blockname}: string\\
  Name of the block.
\end{itemize}

\subsubsection{\texttt{calculate\_block\_centres(\emph{geo})}}

Calculates geometrical centres of all blocks in the grid, based on the specified geometry object \texttt{geo}.

\textbf{Parameters:}
\begin{itemize}
\item \textbf{geo}: \texttt{mulgrid}\\
  Geometry object associated with the grid.
\end{itemize}

\subsubsection{\texttt{check(\emph{fix}=False,\emph{silent}=False)}}

Checks a grid for errors and optionally fixes them.  Errors checked for are: blocks not connected to any other blocks, and blocks with isolated rocktypes (not shared with any neighbouring blocks).  Returns \texttt{True} if no errors were found, and \texttt{False} otherwise.  If \texttt{fix} is \texttt{True}, any identified problems will be fixed.  If \texttt{silent} is \texttt{True}, there is no printout (only really useful if \texttt{fix} is \texttt{True}).

Blocks not connected to any others are fixed by deleting them.  Isolated-rocktype blocks are fixed by assigning them the most popular rocktype of their neighbours.  Blocks with large volumes ($>$ 1E20 m$^3$) are never considered isolated (because they often have a special rocktype, such as an atmosphere one, that their neighbours will never share).

\textbf{Parameters:}
\begin{itemize}
\item \textbf{fix}: Boolean\\
  Whether to fix any problems identified.
\item \textbf{silent}: Boolean\\
  Whether to print out feedback or not.
\end{itemize}

\subsubsection{\texttt{clean\_rocktypes}()}

Deletes any rock types from the grid which are not assigned to any block.

\subsubsection{\texttt{connection\_index(\emph{blocknames})}}

Returns the connection index (in the \texttt{connectionlist} list) of the connection between a specified pair of block names.

\textbf{Parameters:}
\begin{itemize}
\item \textbf{blocknames}: tuple\\
  A pair of block names, each of type string.
\end{itemize}

\subsubsection{\texttt{copy\_connection\_directions(\emph{geo},\emph{grid})}}

Copies the connection permeability directions for horizontal connections from another grid.  It is assumed that both grids have the same column structure, but may have different layer structures.

\textbf{Parameters:}
\begin{itemize}
\item \textbf{geo}: \texttt{mulgrid}\\
  Geometry object associated with the source grid.
\item \textbf{grid}: \texttt{t2grid}\\
  The source grid from which the connection permeability directions are to be copied.
\end{itemize}

\subsubsection{\texttt{delete\_block(\emph{blockname})}}

Deletes a block from the grid.  This also deletes any connections involving the specified block.

\textbf{Parameters:}
\begin{itemize}
\item \textbf{blockname}: string\\
  Name of the block to be deleted from the grid.
\end{itemize}

\subsubsection{\texttt{delete\_connection(\emph{connectionname})}}

Deletes a connection from the grid.

\textbf{Parameters:}
\begin{itemize}
\item \textbf{connectionname}: tuple (of string)\\
  Pair of block names identifying the connection to be deleted from the grid.
\end{itemize}

\subsubsection{\texttt{delete\_rocktype(\emph{rocktypename})}}

Deletes a rock type from the grid.

\textbf{Parameters:}
\begin{itemize}
\item \textbf{rocktypename}: string\\
  Name of the rock type to be deleted from the grid.
\end{itemize}

\subsubsection{\texttt{empty}()}

Empties the grid of all its blocks, rock types and connections.

\subsubsection{\texttt{fromgeo(\emph{geo})}}

Constructs a grid from a \texttt{mulgrid} geometry object.  (Any previous contents of the grid are first emptied.)

\textbf{Parameters:}
\begin{itemize}
\item \textbf{geo}: \texttt{mulgrid}\\
  The \texttt{mulgrid} geometry object (see chapter \ref{mulgrids}).
\end{itemize}

\subsubsection{\texttt{rocktype\_frequency(\emph{rockname})}}

Returns the frequency of use of the rock type with the specified name, i.e. how many blocks are assigned that rock type.

\textbf{Parameters:}
\begin{itemize}
\item \textbf{rockname}: string\\
  Name of the specified rock type.
\end{itemize}

\subsubsection{\texttt{sort\_rocktypes()}}

Sorts the rocktype list into alphabetical order by name.

\subsubsection{\texttt{write\_vtk(\emph{geo}, \emph{filename}, \emph{wells}=False)}}

Writes a \texttt{t2grid} object to a VTK file on disk, for visualisation with VTK, Paraview, Mayavi etc.  The grid is written as an `unstructured grid' VTK object with data arrays defined on cells.  The data arrays written, in addition to the defaults arrays for the associated \texttt{mulgrid} object, are: rock type index, porosity and permeability for each block.  A separate VTK file for the wells in the grid can optionally be written.

\textbf{Parameters:}
\begin{itemize}
\item \textbf{geo}: \texttt{mulgrid}\\
  The \texttt{mulgrid} geometry object associated with the grid.  This is required as the \texttt{t2grid} object does not contain any spatial information, e.g. locations of block vertices.
\item \textbf{filename}: string\\
  Name of the VTK file to be written.  This is also required.
\item \textbf{wells}: Boolean\\
  Set to \texttt{True} if the wells from the \texttt{mulgrid} object are to be written to a separate VTK file.
\end{itemize}

\section{Other objects (\texttt{rocktype}, \texttt{t2block} and \texttt{t2connection})}

A \texttt{t2grid} object contains lists of other types of objects: \texttt{rocktype}, \texttt{t2block} and \texttt{t2connection}.  These classes are described below.

\subsection{\texttt{rocktype} objects}

A \texttt{rocktype} object represents a TOUGH2 rock type.  The properties of a \texttt{rocktype} object, and their default values, are given in Table \ref{tb:rocktype_properties}.

The main familiar properties of a rock type are referred to in a natural way, e.g. the porosity of a rock type \texttt{r} is given by \texttt{r.porosity}.  The permeability property is a 3-element \texttt{np.array}, giving the permeability in each of the three principal axes of the grid, so e.g. the vertical permeability of a rock type \texttt{r} would normally be given by \texttt{r.permeability[2]} (recall that array indices in Python are zero-based, so that the third element has index 2).

Some rock type properties are optional, and only need be specified when the property \texttt{nad} is greater than zero.  An example is the relative permeability and capillarity functions that can be specified for a rock type when \texttt{nad} $\ge$ 2.  The way these functions are specified is described in chapter \ref{datafiles}.

\textbf{Example:}

\begin{verbatim}
r=rocktype(name=`ignim',permeability=[10.e-15,10.e-15,2.e-15],
  specific_heat=850)
\end{verbatim}

declares a rocktype object called \texttt{r} with name `ignim', permeability of 10 mD in the first and second directions and 2 mD in the vertical direction, and specific heat 850 J.kg$^{-1}$.K$^{-1}$.

(Note that when declaring rock types, the permeability can for convenience be specified as a list, which will be converted internally to an \texttt{np.array}.)

\begin{table}
  \begin{center}
    \begin{tabular}{|l|l|p{30mm}|l|}
      \hline
      \textbf{Property} & \textbf{Type} & \textbf{Description} & \textbf{Default}\\
      \hline
      \texttt{capillarity} & dictionary & capillarity function & --\\
      \texttt{compressibility} & float & compressibility & 0 m$^2$.N$^{-1}$\\
      \texttt{conductivity} & float & heat conductivity & 1.5 W.m$^{-1}$.K$^{-1}$\\
      \texttt{density} & float & rock grain density & 2600 kg.m$^{-3}$\\
      \texttt{dry\_conductivity} & float & dry heat conductivity & wet heat conductivity\\
      \texttt{expansivity} & float & expansivity & 0 K$^{-1}$\\
      \texttt{klinkenberg} & float & Klinkenberg parameter & 0 Pa$^{-1}$\\
      \texttt{nad} & integer & number of extra data lines & 0\\
      \texttt{name} & string & rock type name & `dfalt'\\
      \texttt{permeability} & \texttt{np.array} & permeability & \texttt{np.array}([10$^{-15}$]*3) m$^2$\\
      \texttt{porosity} & float & porosity & 0.1\\
      \texttt{relative\_permeability} & dictionary & relative permeability function & --\\
      \texttt{specific\_heat} & float & rock grain specific heat & 900 J.kg$^{-1}$.K$^{-1}$\\
      \texttt{tortuosity} & float & tortuosity factor & 0\\
      \texttt{xkd3} & float & used by EOS7R & 0 m$^{3}$.kg$^{-1}$\\
      \texttt{xkd4} & float & used by EOS7R & 0 m$^{3}$.kg$^{-1}$\\
      \hline
    \end{tabular}
    \caption{Properties of a \texttt{rocktype} object}
    \label{tb:rocktype_properties}
  \end{center}
\end{table}

\subsection{\texttt{t2block} objects}

A \texttt{t2block} object represents a block in a TOUGH2 grid.  The properties of a \texttt{t2block} object are given in Table \ref{tb:t2block_properties}.  These reflect the specifications of a TOUGH2 block as given in a TOUGH2 data file, with the exception of the \texttt{atmosphere}, \texttt{centre}, \texttt{connection\_name}, \texttt{neighbour\_name} and \texttt{num\_connections} properties.

The \texttt{atmosphere} property determines whether the block is to be treated as an atmosphere block.  The \texttt{centre} property can optionally be used to specify the coordinates of the centre of a block.  Block centres are automatically calculated when a \texttt{t2grid} object is constructed from a \texttt{mulgrid} object using the \texttt{fromgeo} method (see section \ref{t2gridmethods}).  The \texttt{connection\_name} property is a set containing the names (as tuples of strings) of all connections involving the block.

A \texttt{t2block} object has no methods.

\begin{table}
  \begin{center}
    \begin{tabular}{|l|l|l|l|}
      \hline
      \textbf{Property} & \textbf{Type} & \textbf{Description}\\
      \hline
      \texttt{ahtx} & float & interface area for heat exchange (TOUGH2 only)\\
      \texttt{atmosphere} & Boolean & whether block is an atmosphere block or not\\
      \texttt{centre} & \texttt{np.array} & block centre (optional)\\
      \texttt{connection\_name} & set & names of connections involving the block\\
      \texttt{nadd} & integer & increment between block numbers in sequence\\
      \texttt{name} & string & block name\\
      \texttt{neighbour\_name} & set & names of neighbouring (connected) blocks\\
      \texttt{nseq} & integer & number of additional blocks in sequence\\
      \texttt{num\_connections} & integer & number of connections containing the block\\
      \texttt{pmx} & float & permeability modifier (TOUGH2 only)\\
      \texttt{rocktype} & \texttt{rocktype} & rock type\\
      \texttt{volume} & float & block volume\\
      \hline
    \end{tabular}
    \caption{Properties of a \texttt{t2block} object}
    \label{tb:t2block_properties}
  \end{center}
\end{table}

\subsection{\texttt{t2connection} objects}

A \texttt{t2connection} object represents a connections between two TOUGH2 blocks.  The properties of a \texttt{t2connnection} object are given in Table \ref{tb:t2connection_properties}.  These correspond to the properties of a connection specified in a TOUGH2 data file.  Note that the \texttt{block} property returns \texttt{t2block} objects, not just the names of the blocks in the connection.  Hence, for example, the volume of the first block in a connection object \texttt{con} is given simply by \texttt{con.block[0].volume}.

A \texttt{t2connection} object has no methods.

\begin{table}
  \begin{center}
    \begin{tabular}{|l|l|l|l|}
      \hline
      \textbf{Property} & \textbf{Type} & \textbf{Description}\\
      \hline
      \texttt{area} & float & connection area\\
      \texttt{block} & \texttt{list} & two-element list of blocks\\
      \texttt{dircos} & float & gravity direction cosine\\
      \texttt{direction} & integer & permeability direction (1, 2, or 3)\\
      \texttt{distance} & \texttt{list} & two-element list of connection distances\\
      \texttt{nad1,nad2} & integer & increments in sequence numbering\\
      \texttt{nseq} & integer & number of additional connections in sequence\\
      \texttt{sigma} & float & radiant emittance factor (TOUGH2 only)\\
      \hline
    \end{tabular}
    \caption{Properties of a \texttt{t2connection} object}
    \label{tb:t2connection_properties}
  \end{center}
\end{table}

\section{Example}

The following piece of Python script creates a rectangular 2-D slice TOUGH2 grid with two rock types, and assigns these rock types to blocks in the grid according to their position along the slice.

\begin{verbatim}
from t2grids import *

geo=mulgrid().rectangular([500]*20,[1000],[100]*20,atmos_type=0,convention=2)
geo.write(`2Dgrd.dat')
grid=t2grid().fromgeo(geo)

grid.add_rocktype(rocktype(`greyw',permeability=[1.e-15]*2+[0.1e-15]))
grid.add_rocktype(rocktype(`fill ',permeability=[15.e-15]*2+[5.e-15]))

for blk in grid.blocklist:
    if 200<=blk.centre[0]<=400: blk.rocktype=grid.rocktype[`fill ']
    else: blk.rocktype=grid.rocktype[`greyw']
\end{verbatim}

The first line just imports the required PyTOUGH library.  (It is not necessary to import the \texttt{mulgrids} library explicitly, because it is used and therefore imported by the \texttt{t2grids} library.)

The second block of code creates a rectangular MULgraph geometry object with 20 columns (each 500 m wide) along the slice and 20 layers (each 100 m thick), writes this to a geometry file on disk, and creates a TOUGH2 grid from it.

Then the two rock types are created, \texttt{`greyw'} and \texttt{`fill '}.  (Note that rock types are expected by TOUGH2 to have names 5 characters long, so it is necessary to add spaces to shorter names.)

The final part assigns the rock types to the grid.  In this case, the blocks in the grid are assigned the \texttt{`fill'} rock type if they are between 200 m and 400 m along the slice.  Blocks outside this region are assigned the \texttt{`greyw'} rock type.
